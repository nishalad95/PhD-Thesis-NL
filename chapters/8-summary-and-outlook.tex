%------------------------------------
%	7. Summary & Outlook
%------------------------------------


\chapter{Conclusion}
\label{chapter-8}

%\subsection{Current Limitations and Refinements}

%\textbf{TODO Here!!}
%Possibly move this to chapter 8 outlook - further work and discussion, current limitations and refinements. These points have been scattered in the main text, this section is mainly a summary of them to present in a easible readable manner. clustering k=1 scenario, pragmatic choices etc, the density of barrel hits and second set of data, retraining certain thresholds and analysing the performance of the algorithm on the ITk geometry to give an indication ... with more time permitting further research and analysis can be conducted to investigate possible improvements. Other considerations and avenues to explore have been discussed below...



\section{Summary}


% 1. Intro & background

The current algorithms employed to reconstruct tracks for High-Energy particle detectors have proven to be very powerful. These methods include seeded track following and the combinatorial KF. However, they are not designed to handle the conditions presented by extreme environments such as the HL-LHC, as these approaches do not scale well as the seed number grows. As discussed in Chapter \ref{chapter-2}, the task of track reconstruction becomes increasingly difficult as the luminosity increases. The collision rate and hence hit occupancy will also grow significantly during future upgrades of the LHC program. Therefore, novel and precise tracking methodologies, as well as efficient use of computing power, are paramount for track reconstruction. The heart of the problem tackles the simultaneous consideration of reducing the amount of CPU usage, and hence the environmental impact, as well as maintaining the ability to reconstruct tracks with minimal loss in efficiency - a trade-off commonly faced within the challenges of track reconstruction and pattern recognition techniques. As such, enhancements in reconstruction algorithms are imperative for the future growth of particle detector experiments. 

The use of ML algorithms for track reconstruction has provided a route to develop methodologies that can save vast amounts of CPU resources. An original approach has been explored in great detail in this thesis utilising GNN architectures for efficient track reconstruction. This process is split into two main components; the first begins with a procedure to construct a graph network from hit-pairs using data from a collision event, this is followed by an algorithm designed to iteratively identify and extract track candidates. 


% 2. Section on graph building and the ML classification predictor algorithm developed…

When constructing a graph network for track reconstruction, an essential aspect to consider is the ability to identify compatible hit connections to build graph edges. Chapter \ref{chapter-4} successfully demonstrates a methodology to achieve this, whereby a ML-based algorithm is developed to predict if a pair of hits belong to the same track. This process is essential for efficient construction of the graph network for track reconstruction, as it successfully reduces the number of fake edges and hence increases the accuracy in predicting compatible hit-pairs. This method also reduces the number of computations required and propagates these benefits to downstream algorithms within the track reconstruction software.

The choice is made to focus on pixel only seeds, being closest to the beamline and interaction point. The pixel detector will have the highest occupancy of hits and naturally this will present a problem for future upgrades as the luminosity of the beams increases. Therefore, reducing the number of fake seeds in this region will be most beneficial. The pixel seeds are used in the early stages of the track seeding procedure and pattern recognition algorithms. By reducing the number of seeds within these early stages of the HLT, this will aid in reducing the CPU usage and propagate the benefits to later stages of the track building pipeline. It is also beneficial to use pixel hits to find compatible hit-pairs and build seeds, as there is a higher degree of precision and granularity in the pixel detector.

The implementation of the ML-based predictor begins with the exploitation of input hit features in its design, namely pixel cluster width and inverse track inclination. The choice of these features are made as the intrinsic cluster position and length will indirectly give information about the inclination angle. The algorithm employs the use of Bayesian analysis and kernel density estimates to discriminate between hit-pair classes. The choice of the Naive Bayes classifier, with a sophisticated likelihood function in the form of a smoothed kernel density, proved to be robust in this data-driven setting and is successful to discriminate between compatible and incompatible doublets. Whereas, alternative approaches to the Naive Bayes classification method, such as the SVM, did not perform as well given the input features as well as due to the nature of the data, indicating that 1-dimensional classifiers were better suited. The implementation of the classifier’s predictions as a LUT is advantageous here in order to reduce computational overheads, which bodes well for use in realistic detectors.

The application of the ML-based classifier was used for pixel seed selection in the ATLAS ID. It has successfully optimised the HLT ID track seeding software for ATLAS Run-3 and beyond, by reducing the number of fake track seeds and provided significant savings in computing resources. The trained predictor in the form of a LUT yields 2.3$\times$ speed-up with minimal loss in efficiency (1.1\%) at $\langle \mu \rangle$ = 80 compared with the standard trigger tracking. 

The developed ML framework is advantageous for many reasons. It is beneficial to have algorithms that allows one to move to different working points of efficiency vs CPU. This provides flexibility when optimising the trigger depending on the LHC conditions (i.e. at varying degrees of pile-up). Another valuable aspect is the ability to interpret probabilities rather than class predictions, as one is able to model trade off concerns between TPR and FPR rates, as well as adjust thresholds accordingly to achieve desired efficiencies. In this work, the TPR (and hence recall) holds higher importance over the precision, as the task is centered around keeping FNs to a minimum in order to keep the prediction of fakes low.

The main loss in track reconstruction efficiency is at large $\eta$. This can be improved by further investigation into the following; better understanding of the hit-pairs located in the transition region, better handling of seeds located within this transition region for classifier training, as well as incorporating greater statistics particularly from these regions to be used as training data.




% 4. GNN algorithm

In this work, particle hits represent graph nodes and compatible hit connections consistent with particle motion models represent graph edge connections. Edge connections can span up to two detector layers apart, in order to compensate for detector inefficiencies such as holes. After the graph network is constructed from a collision event, a procedure is needed to identify and extract track candidates in multi-element silicon detectors. Chapters \ref{chapter-5} and \ref{chapter-6} describe a novel GNN pattern recognition algorithm developed in order to achieve this. The graph network learns local track parameters by iteratively resolving outlier edge connections in order to discover track candidates. 
The main stages of the algorithm consist of a Gaussian mixture reduction stage and neighbourhood information aggregation stage. Each edge in the network is modelled as a Gaussian state vector representing an estimate of the partial track parameters at each node. The network clusters edges together in order to reduce the Gaussian mixture present at each graph node, simultaneously identifying and pruning outlier edge connections. Clustering of state vectors is achieved using the traditional k-means clustering algorithm, whereby the optimal distance of clusterisation is learned via training a SVM. Each node’s neighbourhood of connections is modelled as a single truth track with noisy connections, therefore implemented as $k = 1$. This process proves to be an effective initial step in ambiguity resolution of outlier edges, however a further enhancement would be to extend the k-means implementation to handle intersecting tracks ($k > 1$ scenarios). As edge connections may span up to two detector layers, this means that clustering occurs across layers. Although this implementation works well in this current setup, this feature may need adjusting in further iterations of the algorithm in order to handle complex cases.
The next stage involves the unique use of the Kalman filter, which is an elegant instrument for both information propagation and track extraction. The use of a parabolic track model in the transverse plane works well for state extrapolation and hence is a workable method to identify compatible connections. In addition to this, the linear model proves to be robust when used within the KF for track fitting in the transverse plane. A linear track model alongside OU process noise, is a powerful technique to model low levels of curvature in particle trajectories, implemented using a 3-dimensional filter. In addition to this, application of the GNN-based algorithm to both the transverse and longitudinal planes of detector experiments, allows one to simultaneously consider both projections of particle trajectories.
The iterative methodology of ambiguity resolution in the graph network works well to identify outlier edges. The algorithm yields promising results on both a simple MC toy model, as well as for high track multiplicity regions, particularly the TrackML model for the endcap. A preliminary result of greater than 90\% track reconstruction efficiency is achieved for fully contained tracks within the endcap volume of the TrackML detector, with $p_{\text{T}} >$ 1 GeV. It is promising to see that this methodology provides a high performance on the endcap and has shown to successfully extract track candidates compatible with the particle motion model. This demonstrates that even with suboptimal track reconstruction in this regime, it is possible to make algorithmic advancements to the track reconstruction pipeline to improve the identification of fake seeds and reduce CPU resources, whilst maintaining efficiency. 


\section{Outlook and Future Work}

The GNN-based algorithm presented in this thesis is still within its early stages of development. Further investigation and analysis is needed in order to improve the identification and extraction of track candidates within the barrel region.

One improvement that can be immediately addressed is the use of dynamic hyperparameters, such that the network starts with loose distance thresholds to identify obvious outlier connections, the network proceeds to evaluate its state from the remaining subgraphs and then adjusts the thresholds accordingly. Other such avenues of investigation would include the use of community detection to improve the subdividing of large graph networks into smaller, more manageable subgraphs. In addition, hierarchical clustering methodologies and Cellular Automaton implementations may also improve state information propagation through the network and enhance outlier pruning capabilities.

The most important aspect of this type of algorithm is the graph building stage. If this process begins to build (lose) a significant portion of incompatible (compatible) edge connections, this greatly affects the algorithm’s accuracy and hence its ability to reconstruct tracks. In addition, if an edge is not created in the first place, it cannot be created at a later stage due to the design of the track finding pattern recognition pipeline. However, one may still be able to reconstruct part of the track, as the models follow the kinematics of particle motion. The trade off between computational resources and reconstruction efficiency is an ongoing challenge within track reconstruction and one that has impacted the design and implementation of the algorithms developed in this research. These aspects present fundamental limitations within the scope of this approach and will be of concern for future track finding algorithms as detectors are upgraded.


% 5. conclusions & end
As future upgrades to particle detectors will present reconstruction challenges for silicon tracking detectors in terms of hit occupancy, it is essential that computing resource use is reduced, whilst maintaining tracking capabilities. This provides the motivation for sustainable advancements in algorithmic developments as detectors become increasingly powerful. The findings from the work presented in this thesis show a promising direction that can be taken in ATLAS, as well as other High-Energy particle detector experiments, in order to achieve this goal. 
