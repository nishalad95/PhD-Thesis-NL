Future upgrades to modern High-Energy particle detectors pose considerable challenges for traditional particle track reconstruction methods. As the luminosity and hit-occupancy significantly increase, this presents an enormous strain on CPU resources. Within the past few years, algorithms that operate on Graph Neural Networks (GNN) have shown high degrees of promise. This thesis presents improvements to track reconstruction algorithms, where two novel approaches have been developed. Firstly, a Machine Learning (ML) classifier is designed to predict compatible hit-pairs that belong to the same track, in order to reduce the construction of fake seeds and computational resource use. Secondly, a GNN pattern recognition algorithm has been developed, which utilises an iterative approach to identify and extract track candidates compatible with the particle motion model. This algorithm leverages Gaussian mixture reduction techniques, as well as the Kalman filter as a mechanism for information aggregation and for track extraction. The algorithm is designed to iteratively identify outlier edge connections and improve the precision of track parameters. This procedure differs from other recent approaches, which typically train Multi-Layered Perceptrons (MLPs) and employ deep learning techniques. 

The ML predictor has achieved 2.3$\times$ speed-up with minimal loss in efficiency (1.1\%) with respect to Monte Carlo truth data, and has been deployed in the Run-3 release software of the ATLAS detector’s track seeding algorithm at the LHC. The GNN pattern recognition algorithm for track extraction has been applied to the publicly available dataset designed for the Kaggle TrackML challenge. The algorithm achieves a promising result of greater than 93\% track reconstruction efficiency for fully contained tracks within the Pixel endcap volume of the TrackML detector, with $p_{\text{T}} >$ 1 GeV. Preliminary results of track extraction are discussed for the Pixel barrel region of the TrackML detector, as further investigation and analysis is required. The ultimate aim of this work is to develop a realistic GNN-based algorithm for pattern recognition in order to improve current track finding approaches and as such, can be deployed in future high-luminosity phases of particle detector experiments.
